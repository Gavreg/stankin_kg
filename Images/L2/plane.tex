\documentclass[tikz, class=scrreprt]{standalone}

\usepackage{fp}
\usepackage{tikz}
\usepackage{tikz-3dplot}
\usepackage{amsmath} %математические формулы
\usepackage[e]{esvect}  %Красивая стрелочка вектора


\usetikzlibrary{calc}
\usetikzlibrary{arrows.meta}

\begin{document}
	
	\newcommand{\VARPOINT}[4] 
	{
		
		\coordinate (#1) at (#2 ,#3,#4 );
		\coordinate (#1xy) at (#2 ,#3 ,0);
		\coordinate (#1xz) at (#2 ,0,#4 );
		\coordinate (#1yz) at (0,#3 ,#4 );
		\coordinate (#1x) at (#2 ,0, 0);
		\coordinate (#1y) at (0, #3 ,0);
		\coordinate (#1z) at (0, 0, #4 );
        
        %\FPeval{\x}{#2}
        %\FPeval{\y}{#3}
        %\FPeval{\z}{#4}		
		
		\expandafter\edef\csname #1x\endcsname{\fpeval{#2}}
		\expandafter\edef\csname #1y\endcsname{\fpeval{#3}}
		\expandafter\edef\csname #1z\endcsname{\fpeval{#4}}
		
	}
	
	\newcommand{\VECTOR}[2]
	{
		
		\FPset{\x}{0}
		\FPset{\y}{0}
		\FPset{\z}{0}
		\FPsub{\x}{\csname #2x\endcsname}{\csname #1x\endcsname}
		\FPsub{\y}{\csname #2y\endcsname}{\csname #1y\endcsname}
		\FPsub{\z}{\csname #2z\endcsname}{\csname #1z\endcsname}
		
		
		%\expandafter\VARPOINT {vec#1#2}{\csname #1x}{0}{0}
		\expandafter\VARPOINT {vec#1#2}{\x}{\y}{\z}
	}
	
	\newcommand{\VIPI}[2]
	{
		\def\tmpax{\csname #1x\endcsname}
		\def\tmpay{\csname #1y\endcsname}
		\def\tmpaz{\csname #1z\endcsname}
		
		\def\tmpbx{\csname #2x\endcsname}
		\def\tmpby{\csname #2y\endcsname}
		\def\tmpbz{\csname #2z\endcsname}
		
		\FPeval{\resultx}{\tmpay * \tmpbz - \tmpaz * \tmpby}
		\FPeval{\resulty}{\tmpaz * \tmpbx - \tmpax * \tmpbz}
		\FPeval{\resultz}{\tmpax * \tmpby - \tmpay * \tmpbx}
		
		\expandafter\VARPOINT {vp#1#2}{\resultx}{\resulty}{\resultz}	
		
	}
	
	
	\newcommand{\CALCZ}[4]{ %1-точка 2-n 3c 4-ответ
		
		\edef\tmpx{\csname #1x\endcsname}
		\edef\tmpy{\csname #1y\endcsname}	
		\edef\tmpnx{\csname #2x\endcsname}
		\edef\tmpny{\csname #2y\endcsname}
		\edef\tmpnz{\csname #2z\endcsname}
     	\expandafter\edef\noexpand #4{\fpeval{( #3 - \tmpx * \tmpnx - \tmpy * \tmpny ) /  \tmpnz }}	
		\csname #4\endcsname
	}
	
	
	
	\newcommand{\dotsize}{1pt}
	\newcommand{\PROJECTILELINES}[3] %1 -точка, 2 - стиль линий, 3 стиль кругов
	{
		\draw [#2]  (#1z) -- (#1yz) -- (#1y) -- (#1xy) -- (#1x) -- (#1xz) -- cycle;
		\draw [#2]  (#1) -- (#1xy);     
		\draw [#2]  (#1) -- (#1yz); 
		\draw [#2]  (#1) -- (#1xz); 
		
		\fill [#3] (#1xy) circle [radius=\dotsize];
		\fill [#3] (#1yz) circle [radius=\dotsize];
		\fill [#3] (#1xz) circle [radius=\dotsize];
	}
	
	\newcommand{\PROJECTILELINESXY}[3] %1 -точка, 2 - стиль линий, 3 стиль кругов
	{
		\draw [#2]  (#1) -- (#1xy);     		
		\fill [#3] (#1xy) circle [radius=\dotsize];

	}
	\def\frame{100}
	\foreach \frame in {0,1,...,2}
	{
	\tdplotsetmaincoords{70}{100}
	\begin{tikzpicture}[scale=1, tdplot_main_coords]
		
		
		%\draw[step=.5cm,black!10,very thin] (-1.5,-1.5) grid (5,5);
		
		%\draw[-latex, thick] (-1,0) -- (5,0);
		%\draw[-latex, thick] (0,-1) -- (0,5);
		
		\coordinate (O) at (0,0);
		
		%\draw (O) node [left,yshift=5pt] {$O$};
		
		% A(1,1)   B(3,1)
		% C(1,3)  D(3,3) 
		% n=(2,1,3)
		%
		%
		%
		%
		%
		
		\VARPOINT{N}{1}{1}{3};
		
		
		
		\coordinate (i) at (1,0,0);
		\coordinate (j) at (0,1,0);
		\coordinate (k) at (0,0,1);
		
		\draw [-Latex, red] (O) -- ($5*(i)$);
		\draw [-Latex, green] (O) -- ($4*(j)$);
		\draw [-Latex, blue] (O) -- ($4*(k)$);
		
		
		%\coordinate (A) at (1,1,3);
		%\coordinate (B) at (0.5,4,4);
		%\coordinate (C) at (4,1.5);
		
		\VARPOINT{tA}{1}{1}{0};  % 2 4 2		
		\VARPOINT{tB}{3}{1}{0};
		\VARPOINT{tC}{1}{3}{0};
		\VARPOINT{tD}{3}{3}{0};
		\VARPOINT{tE}{2}{2.5}{0};
		\VARPOINT{tP}{1.5}{1.4}{0};
		
				
	    \CALCZ{tA}{N}{8}{\zA}
	    \CALCZ{tB}{N}{8}{\zB}
	    \CALCZ{tC}{N}{8}{\zC}
	    \CALCZ{tD}{N}{8}{\zD}	    
	    \CALCZ{tE}{N}{8}{\zE}
	    \CALCZ{tP}{N}{8}{\zP}
	    
	    \VARPOINT{A}{\tAx}{\tAy}{\zA};  % 2 4 2		
	    \VARPOINT{B}{\tDx}{\tBy}{\zB};
	    \VARPOINT{C}{\tCx}{\tCy}{\zC};
	    \VARPOINT{D}{\tDx}{\tDy}{\zD};
	    \VARPOINT{E}{\tEx}{\tEy}{\zE};
	    \VARPOINT{P}{\tPx}{\tPy}{\zP};
	    
	    
	    
	    \renewcommand{\dotsize}{0.5pt}
	    \PROJECTILELINESXY{A}{dashed, very thin, black!50}{black!50}
	    \PROJECTILELINESXY{B}{dashed, very thin, black!50}{black!50}
	    \PROJECTILELINESXY{C}{dashed, very thin, black!50}{black!50}
	    \PROJECTILELINESXY{D}{dashed, very thin, black!50}{black!50}
		
		
		\draw[-Latex] (O) -- +(N);
		\draw[dashed, very thin, black!50] ;
		
		
		
		
		\draw[-Latex] (O) -- (E);
        
		
		
		

		
		\ifthenelse{\frame = 0}{
				\filldraw[red, fill=red!50, opacity=0.5] (A) -- (B) -- (D) -- (C) -- cycle;
			}{}

		
		

		\ifthenelse{\frame > 0}{
		    
		    \draw[-Latex, densely dashed] (O) -- (P);
		    \filldraw[red, fill=red!50, opacity=0.5] (A) -- (B) -- (D) -- (C) -- cycle;
		    
		 	\draw (P) node [above right] {$\vv{p}$};
		    \fill [] (P) circle [radius=1pt];	
		
		}{}
		
		\ifthenelse{\frame > 1}{
			
			\draw[-Latex, densely dashed] (P) -- (E) node [midway, sloped, below]{\tiny $\vv{p}-\vv{a}$};
			
		}{}
		
		
		\fill [] (E) circle [radius=1pt];
        \draw (N) node [right] {$\vv{n}$};		
        \draw (E) node [right] {$\vv{a}$};
        
		\draw (C) node [above left] {$\alpha$};	

		
		
		
	\end{tikzpicture}}
	
\end{document}