


\mode<article>{
	
	\usepackage{hyperref}
	
}
\mode<presentation>{
	
	\usetheme{Antibes}
	\usefonttheme{professionalfonts} 
	\usefonttheme{serif} % default family is serif
	
	%\usecolortheme{spruce} %зеленая, плохой цвет в заголовках 
	%\usecolortheme{albatross} %синяя, пхоло виден черный цвет
	
}

\newcommand{\MP}[1]{\mode<presentation>{#1} }
\newcommand{\MA}[1]{\mode<article>{#1} }

\newcommand{\ABS}[1]{\left| #1 \right|}
%\newcommand{\ABS}[1]{\mid #1 \mid}

\newcommand{\HREF}[2]{{\color{blue}\underline{\href{#1}{#2}}}}

\setbeamertemplate{caption}[numbered]


%\usepackage[T2A]{fontenc}
%\usepackage[utf8]{inputenc}
%\usepackage[russian]{babel}
%\usepackage{amsmath} %математические формулы



\usepackage{ifthen}

\usepackage{tikz}
\usetikzlibrary{arrows.meta}

\usepackage{fp}
\usepackage{tikz-3dplot}
\usepackage{environ}
\usepackage{animate}




\usepackage{xcolor}
%\usepackage[left=20mm,right=20mm,top=20mm,bottom=20mm,a4paper]{geometry} %поля

\usepackage{amsmath} %математические формулы


\usepackage[e]{esvect}  %Красивая стрелочка вектора
%\let\oldvv\vv
\newcommand{\VV}[1]{\vv{#1\mathstrut}}



\usepackage{graphicx} %работа с каритнками


\usepackage{multimedia}

%Для XeLatex/+
\usepackage{polyglossia}
\setdefaultlanguage{russian}
\setotherlanguage{english}
\setkeys{russian}{babelshorthands=true}


\usepackage{fontspec}

\setmainfont{Times New Roman} [Script=Cyrillic, Mapping=tex-text,]
\setsansfont{Arial} [Script=Cyrillic, Mapping=tex-text,]
\setmonofont{Courier New} [Script=Cyrillic, Mapping=tex-text,]



\usepackage{unicode-math}
%\setmathfont{TeX Gyre Termes Math}

%\setmainfont{CMU Serif}[Script=Cyrillic, Mapping=tex-text,]
%\setsansfont{CMU Sans Serif}[Script=Cyrillic, Mapping=tex-text,]
%\setmonofont{CMU Typewriter Text}[Script=Cyrillic, Mapping=tex-text,]


%-----------------


%\usepackage{caption}
%\DeclareCaptionLabelSeparator{dot}{~---~}            %Разделитель номер рисунка
%\captionsetup[figure]{justification=centering,labelsep=dot, format=plain}                        %Подпись рис. центр
%\captionsetup[table]{justification=raggedleft,labelsep=dot, format=plain, singlelinecheck=false} %Подпись табл. слева
%\captionsetup[lstlisting]{justification=raggedleft,labelsep=dot, format=plain, singlelinecheck=false}                     %Подпись рис. центр

\usepackage{indentfirst} %отступ первой строки


\usepackage[svgnames]{xcolor}


%\usepackage{showframe}


%\usepackage{tikz}

%\usepackage[hidelinks]{hyperref}%ссылки внутри документа \ref


\setlength\abovecaptionskip{-2pt}
%\setlength\belowcaptionskip{-14pt}

\setbeamerfont{caption}{size=\scriptsize}


\def\sectionname{Раздел}
\def\subsectionname{Подраздел}


\newcommand{\TC}[3]
{
	
	
	\begin{columns}
		\begin{column}{#1\textwidth}
			#2
		\end{column}
		\begin{column}{\fpeval{1-#1}\textwidth}
			#3
		\end{column}
	\end{columns}
}

\newcommand{\TCT}[3]
{
	
	\begin{columns}[T]
		\begin{column}{#1\textwidth}
			#2
		\end{column}
		\begin{column}{\fpeval{1-#1}\textwidth}
			#3
		\end{column}
	\end{columns}
}


\newcommand{\FRAME}[2]{
	\begin{frame}
		\frametitle{#1}
		#2
	\end{frame}
}

\newcommand{\FIG}[3]
{
	\begin{figure}
		\centering
		\includegraphics[width=#3]{#1}
		\caption{#2}
	\end{figure}
}

\newcommand{\vect}[1]{\overrightarrow{#1}}


\usepackage{newfile}

\edef\LectionNumber{0}

\let\oldsection\section
\let\oldsubsection\subsection


\AtBeginDocument
{
	\newoutputstream{CONTENT}
	\openoutputfile{\LectionNumber .gvr}{CONTENT}
	
	\expandafter\addtostream{CONTENT}{\noindent\textbf{\Large\inserttitle}\unexpanded{\setcounter{SEC}{0}}\par}
}

\renewcommand{\section}[1]{
	\oldsection{#1}
	\expandafter\addtostream{CONTENT}{\noindent\hspace{2ex}\unexpanded{\hbox{\large\stepcounter{SEC}\theSEC ~ #1}}\par}
}

\renewcommand{\subsection}[1]{
	\oldsubsection{#1}
	\expandafter\addtostream{CONTENT}{\noindent\hspace{6ex}\unexpanded{\stepcounter{SUB}\theSUB ~ #1}\par}
}

%\renewcommand{\section}[1]{\MMM{#1}}

%\edef\subsection#1
{
	%\noexpand\subsection{#1}
	%
}


\author{Гаврилов Андрей Геннадьевич}
\institute{Кафедра Информационных технологий и вычислительных систем \\МГТУ~<<СТАНКИН>>}
\lecture{История компьютерной графики}{kghistory}\subtitle{Компьютерная графика}

